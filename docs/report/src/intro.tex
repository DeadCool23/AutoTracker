\begin{center}
    \textbf{ВВЕДЕНИЕ}
\end{center}
\addcontentsline{toc}{chapter}{ВВЕДЕНИЕ}

Отслеживание маршрутов автомобилей с использованием данных с камер представляет собой важную задачу, направленную на повышение безопасности дорожного движения, предотвращение краж транспортных средств и других правонарушений. Для эффективного сбора, хранения и анализа информации, поступающей с камер, требуется разработка базы данных и приложения, обеспечивающего доступ к ней. Такая система позволит отслеживать службам безопасности маршруты подозрительных автомобилей, а также даст возможность простым пользователям просматривать маршруты своих автомобилей в случае угона. 

\textbf{Цель работы}: разработка базы данных для сервиса отслеживания маршрутов автомобилей по данным с камер, а также приложения для доступа к ней.

Для достижения поставленной цели необходимо решить следующие
задачи:

\begin{itemize}
    \item[---] проанализировать существующие решения и предметную область;
    \item[---] сформулировать требования к разрабатываемой базе данных и приложению;
    \item[---] спроектировать сущности и ограничения базы данных;
    \item[---] выбрать средства реализации базы данных и программного обеспечения;
    \item[---] реализовать базу данных и программное обеспечения для доступа к ней;
    \item[---] провести исследования зависимости времени вставки данных в базу данных от метода добавления данных.
\end{itemize}